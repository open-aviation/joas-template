\documentclass[
  manuscript=proceedings,  %% article (default), rescience, data, software, proceedings, poster
  layout=preprint,  %% preprint (for submission) or publish (for publisher only)
  year=20xx,
  volume=x,
]{extra/joas}

\doi{xx.xxxxx/joas.xxxx.xxxx}

% \conference{} command is only used for proceedings
\conference{The 12th OpenSky Symposium}

\received {1 April 20xx}
\revised  {1 May 20xx}
\accepted {10 May 20xx}
\published{20 May 20xx}

\editor{Editor Name}

\reviewers{First Reviewer, Second Reviewer, Third Reviewer}


% --- blew is the area for authors ---

% remove the following two packages, and delete all \blindtext commands
\usepackage[english]{babel} 
\usepackage{blindtext}


% specify the .bib file for references
\addbibresource{reference.bib} 


% Make sure your article tile is within 12 words
\title{Using a concise title for your article}

\author{First Author \orcid{0000-0000-0000-0000}}
\affiliation{Institution-1, City, Country}
\email{correspondence@email.domain}

\author{Second Author \orcid{0000-0000-0000-0000}}
\affiliation{Institution-2, City, Country}

\author{Third Author}
\alsoaffiliation{Institution-1, City, Country}
\affiliation{Institution-3, City, Country}


% maximum five keywords
\keywords{keyword; keyword-two; keyword number three} 

% Important: don't over use abbreviations. Only use abbreviation if the term is used more than ten times throughout the paper. Otherwise, write them in full.
\abbreviations{
    JOAS: Journal of Open Aviation Science, 
    ATM: Air Traffic Management
}


\begin{document}

\begin{abstract}
  An abstract summarizes in one paragraph with 300 words or less, the major aspects of the entire paper. They often include: 1) the overall purpose of the study and the research problem you investigated; 2) the basic design of you research approach; 3) major findings as a result of your analysis; and, 4) a brief summary of your interpretations and conclusions. 
\end{abstract}


\section{Introduction}

\blindtext 
open data in science \cite{murray2008open}. 


\blindtext [2]


\section{Method}

\subsection{Method part 1}

\blindtext The end result is in Figure \ref{fig:logo}.

\begin{figure}[ht!]
  \centering
  \includegraphics[width=0.45\textwidth]{example-image}
  \caption{An Example Figure}
  \label{fig:logo}
\end{figure}

\blindtext\footnote{This is how a footnote works.}

\subsection{Method part 2}

\subsubsection{Method part 2-1}

\blindtext Reference to Equation \ref{eq:cauchy_momentum}.

\begin{equation} \label{eq:cauchy_momentum}
\rho\frac{\mathrm{D} \mathbf{u}}{\mathrm{D} t} = - \nabla p + \nabla \cdot \boldsymbol \tau + \rho\,\mathbf{g}
\end{equation}


\subsubsection{Method part 2-2}

\blindtext Table \ref{tb:example_table} shows an example.

\begin{table}[H]
  \centering
  \small
  \caption{Example table}
  \label{tb:example_table}
  \begin{tabular}{lll}
  \toprule
  \textbf{Parameter} & \textbf{Notation} & \textbf{Remarks} \\
  \midrule
  name & - & engine common identifier \\
  manufacture & - & name of the manufacture  \\
  bpr & $\lambda$ & bypass ratio \\
  pr & - & pressure ratio \\
  thrust & $T_0$ & maximum static thrust\\
  \bottomrule
  \end{tabular}
\end{table}

\blindtext


\section{Discussions}

\paragraph{Paragraph title} This is the paragraph with title if you want to use such function in the paper. \blindtext


\section{Conclusion}

\blindtext


\appendix

\section{Supplementary figures}
\blindtext

\section{Supplementary tables}
\blindtext


\section*{Acknowledgement}
Include your acknowledgement in this section.

% Author contributions (CRediT) are mandatory for all papers with more than one author
\section*{Author contributions}
  If the paper has more than one author, the CRediT section must be included. See example usage on \url{https://casrai.org/credit/}

  \begin{itemize}
    \item First Author: Conceptualization, Data Curation, Formal Analysis, Funding Acquisition, Investigation, Methodology, Project Administration, Resources, Software, Supervision, Validation, Visualization, Writing (Original Draft), Writing (Review and Editing)
    \item Second Author: Data curation, Writing- Original draft
    \item Third Author: Visualization, Investigation
  \end{itemize}


\section*{Funding statement}
When applicable, please specify the funding information for this research.


% Data statement is mandatory for all papers
\section*{Open data statement}
DOI and short description to supplementary data.

% reproducibility statement is mandatory for all papers
\section*{Reproducibility statement}
Information on how to reproduce this research, including access to 1) source code related the research, 2) source code for the figures, 3) source code / data for the tables when applicable.



\printbibliography



\end{document}
