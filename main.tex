\documentclass[
  manuscript=article,  %% article (default), rescience, data, software, proceedings, poster
  layout=preprint,  %% preprint (for submission) or publish (for publisher only)
  year=20xx,
  volume=x,
]{extra/joas}

\doi{xx.xxxxx/joas.xxxx.xxxx}

% \conference{} command is only used for proceedings
\conference{Conference Title}

\received {1 April 20xx}
\revised  {1 May 20xx}
\accepted {10 May 20xx}
\published{20 May 20xx}

\editor{Editor Name}

\reviewers{First Reviewer, Second Reviewer, Third Reviewer}

% specify the .bib file for references
\addbibresource{reference.bib} 


% --- blew is the area for authors ---

% Title within 12 words
\title{User Guide for Journal of Open Aviation Science Template}

\author{First Author \orcid{0000-0000-0000-0000}}
\affiliation{Institution-1, City, Country}
\email{correspondence@email.domain}

\author{Second Author \orcid{0000-0000-0000-0000}}
\affiliation{Institution-2, City, Country}

\author{Third Author}
\alsoaffiliation{Institution-1, City, Country}
\affiliation{Institution-3, City, Country}


% maximum five keywords
\keywords{keyword; keyword-two; keyword number three} 

% Important: don't over use abbreviations. Only use abbreviation if the term is used more than ten times throughout the paper. Otherwise, write them in full.
\abbreviations{
    JOAS: Journal of Open Aviation Science, % remove this in your paper!
    ATM: Air Traffic Management % remove this in your paper!
}


\begin{document}

\begin{abstract}
  An abstract summarizes in one paragraph with 300 words or less, the major aspects of the entire paper. They often include: 1) the overall purpose of the study and the research problem you investigated; 2) the basic design of you research approach; 3) major findings as a result of your analysis; and, 4) a brief summary of your interpretations and conclusions. 
\end{abstract}



\section{User guide for this template}

\textcolor{red}{DO NOT submit papers containing latex error messages}. If you see any error messages, please fix them before submission. Please read the following guidelines carefully.

The copy editor is Junzi alone for the moment, any time saved for him is time saved for you :).

\subsection{Title}
The title should be in Title Case. Keep it concise and informative, ideally within 12 words. Avoid abbreviations in the title unless they are widely recognized.

\subsection{Single main file}
\textcolor{red}{DO NOT use external .tex files}, like \verb|\input{}| or \verb|\include{}|. Place all content in \texttt{main.tex}. 

\subsection{Keep the original file names}
\textcolor{red}{DO NOT rename filenames} including \texttt{main.tex}, the \texttt{figures} folder, or \texttt{reference.bib}, all to ensure compatibility with the automated copyediting process.

\subsection{References}
Add your bibliography entries to \texttt{reference.bib} and cite them using \verb|\cite{}|. Here is an example of a citation on open data \cite{murray2008open}. Another example on diamond open access \cite{fuchs2013diamond}. Sine many of you are using the OpenSky data, here is another example \cite{schafer2014bringing}.

\subsection{Footnotes}
Use footnotes sparingly. They should be used for additional information that is not essential to the main text. Use the \verb|\footnote{}| command to create footnotes.\footnote{This is an example footnote works.}

\subsection{Tables}
Use the standard \texttt{tabular} environment. \textcolor{red}{Avoid custom or complex table designed} to ensure compatibility for the web version. Table \ref{tb:example_table} shows an example that always works.

\begin{table}[htbp!]
  \centering
  \small
  \caption{Example table}
  \label{tb:example_table}
  \begin{tabular}{lll}
  \toprule
  \textbf{Parameter} & \textbf{Notation} & \textbf{Remarks} \\
  \midrule
  name & - & engine common identifier \\
  manufacture & - & name of the manufacture  \\
  bpr & $\lambda$ & bypass ratio \\
  pr & - & pressure ratio \\
  thrust & $T_0$ & maximum static thrust\\
  \bottomrule
  \end{tabular}
\end{table}

\subsection{Figures}
Store all figures in the one \textbf{figures} folder. Use concise, space-free and lowercase filenames.

Ensure that the figure is in a compatible format (.png or .pdf) and is appropriately sized (not too large or too small). Figure~\ref{fig:example} shows an example of how to include a figure. 

\begin{figure}[htbp!]
  \centering
  \includegraphics[width=0.4\textwidth]{example-image}
  \caption{An Example Figure}
  \label{fig:example}
\end{figure}

If you need subfigures, \textcolor{red}{DO NOT use the \texttt{subfloat} package}. You are recommended to use the \texttt{subfigure} package instead. Figure~\ref{fig:subfig_example} shows an example of how to use subfigures.

\begin{figure}[htbp!]
  \centering
  \subfigure[First subfigure]{\includegraphics[width=0.3\textwidth]{example-image-a}}
  \hspace{0.2cm}
  \subfigure[Second subfigure]{\includegraphics[width=0.3\textwidth]{example-image-b}}
  \caption{An Example of Subfigures using the subfigure package}
  \label{fig:subfig_example}
\end{figure}

Attentiion: in your text, use \verb|Figure \ref{fig:example}|, NOT \verb|Fig. \ref{fig:example}|.


\subsection{Equations}
Use the \verb|equation| environment for numbered equations. You must avoid customized variable names. Otherwise, the HTML version will not be generated properly.

For example, Equation \ref{eq:cauchy_momentum} shows an example equation.

\begin{equation} \label{eq:cauchy_momentum}
\rho\frac{\mathrm{D} \mathbf{u}}{\mathrm{D} t} = - \nabla p + \nabla \cdot \boldsymbol \tau + \rho\,\mathbf{g}
\end{equation}

\subsection{Abbreviations}
Use the \verb|\abbreviations{}| command to define abbreviations. Only use abbreviations if the term is used more than ten times throughout the paper. Otherwise, write them in full.


\section{Sections}

Organize your paper using standard sectioning commands (\verb|\section|, \verb|\subsection|, etc.).

Some standard sections are:

\begin{itemize}
  \item Introduction
  \item Methods
  \item Results
  \item Discussion
  \item Conclusion
\end{itemize}

You can add or remove sections as needed.

Use \verb|Appendix| for supplementary material. The appendix should be used for additional information that is not essential to the main text but may be useful for some readers. Remove this section if you do not have any supplementary material.

\appendix

\section{Supplementary figures}

\section{Supplementary tables}


\section*{Acknowledgement}
Include your acknowledgement in this section.

% Author contributions (CRediT) are mandatory for all papers with more than one author
\section*{Author contributions}
If the paper has more than one author, the CRediT section must be included. See example usage on \url{https://casrai.org/credit/}

\begin{itemize}
  \item First Author: Conceptualization, Data Curation, Formal Analysis, Funding Acquisition, Investigation, Methodology, Project Administration, Resources, Software, Supervision, Validation, Visualization, Writing (Original Draft), Writing (Review and Editing)
  \item Second Author: Data curation, Writing- Original draft
  \item Third Author: Visualization, Investigation
\end{itemize}


\section*{Funding statement}
When applicable, please specify the funding information for this research.


% Data statement is mandatory for all papers
\section*{Open data statement}
\textcolor{red}{Mandatory section!}

Include DOI and short description to supplementary data.

% reproducibility statement is mandatory for all papers
\section*{Reproducibility statement}
\textcolor{red}{Mandatory section!}

Information on how to reproduce this research, including access to 1) source code related the research, 2) source code for the figures, 3) source code / data for the tables when applicable.



\printbibliography



\end{document}
