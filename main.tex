\documentclass[
  manuscript=article,  %% article (default), rescience, data, or software
  layout=preprint,  %% preprint (for submission) or publish (for publisher only)
  year=20xx,
  volume=x,
]{joas}

\doi{10.74800/joas.x.xxxx}

\received {1 April 2022}
\revised  {1 May 2022}
\accepted {10 May 2022}
\published{20 May 2022}


% --- blew is the area for authors ---


\usepackage[english]{babel}
\usepackage{blindtext}

\addbibresource{reference.bib} % specify the .bib file for references

\title{Creating an open community to support reusable and reproducible aviation science}

\author{Junzi Sun}
\affiliation{Joint first authors}
\alsoaffiliation{Faculty of Aerospace Engineering, Delft University of Technology, Netherlands}
\email{j.sun-1@tudelft.nl}

\author{Xavier Olive}
\affiliation{Joint first authors}
\alsoaffiliation{Office National d'Etudes et de Recherches Aérospatiales, France}

\author{Martin Strohmeier}
\affiliation{OpenSky Network, Switzerland}
\alsoaffiliation{University of Oxford, UK}

\author{Enrico Spinielli}
\affiliation{EUROCONTROL, Belgium}

\author{Tejas Puranik}
\affiliation{NASA Ames Research Center, USA}

\author{Nàme with mañý diačriticś (font check)}
\affiliation{University of Syldavia}


%% first word caps
\keywords{Aviation; reproducibility; transparency; data; software; re-science} 

\abbreviations{
    ATM: Air Traffic Managment, 
    ADS-B: Automatic Dependent Surveillance--Broadcast
}


\begin{document}

\begin{abstract}
  \blindtext
\end{abstract}


\section{Introduction}

\blindtext Some example open open-source research data \cite{schafer2014bringing} and tools \cite{olive2019traffic,sun2020openap}. The end result is in Figure \ref{fig:logo}: nice, isn't it?


\blindtext[2]

\blindtext See also Table \ref{tb:eng_perf_params}.


\section{Method}

Callsigns are nice in tt mode \texttt{AFR88HH} and $\rho = 1$ inline

\paragraph{Paragraph title} \blindtext

\subsection{Method 1}

\blindtext

\begin{figure}[ht!]
  \centering
  \includegraphics[width=0.6\textwidth]{joas-logo.pdf}
  \caption{JOAS Logo}
  \label{fig:logo}
\end{figure}

\blindtext\footnote{This is how a footnote works.}

\subsection{Method 2}

\subsubsection{Method 2-1}

\blindtext

\begin{equation}
  E = m c^2
\end{equation}

\blindtext

\begin{table}[H]
  \centering
  \small
  \caption{Engine performance parameters}
  \label{tb:eng_perf_params}
  \begin{tabular}{lll}
  \toprule
  \textbf{Parameter} & \textbf{Notation} & \textbf{Remarks} \\
  \midrule
  name & - & engine common identifier \\
  manufacture & - & -  \\
  bpr & $\lambda$ & bypass ratio \\
  pr & - & pressure ratio \\
  max\_thrust & $T_0$ & maximum static thrust, sea level (unit: N) \\
  fuel\_c3 & $C_\mathrm{ff3}$ & fuel flow coefficient, 3rd order term (unit: kg/s) \\
  fuel\_c2 & $C_\mathrm{ff2}$ & fuel flow coefficient, 2nd order term (unit: kg/s) \\
  fuel\_c1 & $C_\mathrm{ff1}$ & fuel flow coefficient, 1st order term (unit: kg/s) \\
  cruise\_thrust & $T_\mathrm{cr}$ & thrust at the top of climb (unit: N) \\
  cruise\_mach & $M_\mathrm{cr}$ & cruise Mach number for the thrust condition \\
  cruise\_alt & $h_\mathrm{cr}$ & cruise Mach altitude for the thrust condition (unit: ft) \\
  \bottomrule
  \end{tabular}
\end{table}

\blindtext

\section{Conclusion}

\Blindtext


\begin{acknowledgement}
\blindtext
\end{acknowledgement}


\printbibliography

\appendix

\section{Supplementary figures}
\blindtext

\section{Supplementary data tables}
\blindtext

% The following reproducibility statement is mandatory for all papers
\begin{reproduce}
Information on how to access supplementary data to reproduce this research, including access to open data set, source code for the research, source code for the figures, etc.
\end{reproduce}


\end{document}
