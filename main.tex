\documentclass[
  manuscript=proceeding,  %% article (default), rescience, data, software, proceeding
  layout=preprint,  %% preprint (for submission) or publish (for publisher only)
  year=20xx,
  volume=x,
]{extra/joas}

\doi{10.74800/joas.x.xxxx}

% \conference{} only used for proceedings
\conference{The 11th OpenSky Symposium}

\received {1 April 2022}
\revised  {1 May 2022}
\accepted {10 May 2022}
\published{20 May 2022}
\editor{Editor Name}
\reviewers{First Reviewer, Second Reviewer, Third Reviewer}


% --- blew is the area for authors ---


\usepackage[english]{babel}
\usepackage{blindtext}

% specify the .bib file for references
\addbibresource{reference.bib} 

% Make sure your article tile is within 12 words
\title{Using a concise title for your article}

\author{First Author}
\affiliation{Institution-1, City, Country}
\email{correspondence@email.domain}

\author{Second Author}
\affiliation{Institution-2, City, Country}

\author{Third Author}
\alsoaffiliation{Institution-1, City, Country}
\affiliation{Institution-3, City, Country}


% maximum five keywords
\keywords{keyword; keyword one; keywork two} 

% Important, only index abbreviation if a term contains more than two words, and the term is used more than ten times throughout the paper. Otherwise, spell them out in full in the paper.
\abbreviations{
    JOAS: Journal of Open Aviation Science, 
    ATM: Air Traffic Managment
}


\begin{document}

\begin{abstract}
  An abstract summarizes in one paragraph with 300 words or less, the major aspects of the entire paper. They often include: 1) the overall purpose of the study and the research problem you investigated; 2) the basic design of you research approach; 3) major findings as a result of your analysis; and, 4) a brief summary of your interpretations and conclusions. 
\end{abstract}


\section{Introduction}

\blindtext 
Some example open open-source research data \cite{schafer2014bringing} and tools \cite{olive2019traffic}. 


\blindtext [2]


\section{Method}

\subsection{Method part 1}

\blindtext The end result is in Figure \ref{fig:logo}.

\begin{figure}[ht!]
  \centering
  \includegraphics[width=0.6\textwidth]{extra/joas-logo.pdf}
  \caption{JOAS Logo}
  \label{fig:logo}
\end{figure}

\blindtext\footnote{This is how a footnote works.}

\subsection{Method part 2}

\subsubsection{Method part 2-1}

\blindtext Reference to Equation \ref{eq:cauchy_momentum}.

\begin{equation} \label{eq:cauchy_momentum}
\rho\frac{\mathrm{D} \mathbf{u}}{\mathrm{D} t} = - \nabla p + \nabla \cdot \boldsymbol \tau + \rho\,\mathbf{g}
\end{equation}


\subsubsection{Method part 2-2}

\blindtext Table \ref{tb:example_table} shows an example.

\begin{table}[H]
  \centering
  \small
  \caption{Example table}
  \label{tb:example_table}
  \begin{tabular}{lll}
  \toprule
  \textbf{Parameter} & \textbf{Notation} & \textbf{Remarks} \\
  \midrule
  name & - & engine common identifier \\
  manufacture & - & name of the manufacture  \\
  bpr & $\lambda$ & bypass ratio \\
  pr & - & pressure ratio \\
  thrust & $T_0$ & maximum static thrust\\
  \bottomrule
  \end{tabular}
\end{table}

\blindtext


\section{Discussions}

\paragraph{Paragraph title} This is the paragraph with title if you want to use such function in the paper. \blindtext


\section{Conclusion}

\blindtext


\appendix

\section{Supplementary figures}
\blindtext

\section{Supplementary tables}
\blindtext


\begin{acknowledgement}
  Include your acknowledgement in this section.
\end{acknowledgement}

% Author contributions (CRediT) are mandatory for all papers with more than one author
\begin{credit}
  If the paper has more than one author, the CRediT section must be included. See example usage on \url{https://casrai.org/credit/}

  \begin{itemize}
    \item First Author: Conceptualization, Methodology, Software, Writing- Original draft
    \item Second Author: Data curation, Writing- Original draft
    \item Third Author: Visualization, Investigation
  \end{itemize}
\end{credit}


\begin{funding}
  When applicable, please specify the funding information for this research.
\end{funding}


% Data statement is mandatory for all papers
\begin{opendata}
  DOI and short description to supplementary data.
\end{opendata}

% reproducibility statement is mandatory for all papers
\begin{reproduce}
Information on how to reproduce this research, including access to 1) source code related the research, 2) source code for the figures, 3) source code / data for the tables when applicable.
\end{reproduce}



\printbibliography



\end{document}
